\chapter{绪论}
\label{introduction}

\section{背景介绍}
在凝聚态物理领域,探索与分类各种物质的状态是一项非常引人注目的重要主题。整数和分数量子霍尔效(quantum Hall effect, QHE)应更是其中的研究焦点。众所周知,在二维电子气中,施加一个均匀的外磁场会产生高度简并的朗道能级。当相互作用的粒子分数填充在朗道能级上时,体系便有可能出现分数量子霍尔效应。迄今为止,由于只有在强磁场和极低温的二维电子气中观测到该效应,这一效应可能的现实应用受到了极大的限制。另一方面,在无外加均匀磁场的晶格体系中实现整数和分数量子霍尔效应引起了物理研究者极大的兴趣。新近提出的拓扑平坦能带满足了探索这一非朗道能级上的分数化现象的基本条件。拓扑平坦能带至少含有一支陈数非零的、平坦度较高的拓扑能带,这可以近似看作是连续朗道能级的晶格体系对应。最近在拓扑平带模型上开展的一些数值工作,系统地研究了其中的强关联效应,牢固地确立了在无朗道能级情形下的阿贝尔型和非阿贝尔型分数量子霍尔效应。新发现的分数量子霍尔效应不同与传统朗道能级上的连续型分数量子霍尔效应,无需外加磁场,有较大的特征能隙,因此有可能在常温下存在。不需要朗道能级,不能用常规的Laughlin波函数描述。这些无外加磁场、无朗道能级的分数化现象,定义了一类新的分数拓扑相,也称为分数陈绝缘体,其中的分数量子霍尔效应也称为分数量子反常霍尔效应。由于其可能的现实应用,该领域在近期引起了物理研究者极大地兴趣。

一些新的研究手段,例如基于Wannier表象的模型波函数和赝势法、投影密度算符代数、部分子(parton)波函数构造、绝热连通路径等方法被快速发展起来,以进一步理解这些分数量子反常霍尔态。多个研究小组提出了其他的拓扑平带模型以及可能的材料实现方案。最近的数值工作又发现了高陈数(high Chern number)的拓扑平带上的分数量子反常霍尔效应,由于每一条朗道能级的陈数总是1,这些分数化效应并没有直接的朗道能级上的连续对应。另外还观测到了非阿贝尔型(Non-abelian)以及叠代(hierarchy)分数量子反常霍尔态。


\section{关于本文}
尽管此前的研究已经牢固地确定了分数量子反常霍尔态(FCI/FQAH)的各种拓扑性质,但是边缘激发的研究在原则上可以提供另一个揭示体态拓扑序的窗口。考虑到将来可能会首先在光晶格冷原子体系实现分数量子反常霍尔态(FCI/FQAH),边缘激发也可以提供一个切实可行的实验探测手段。在实验方面,已经提出了一些非常可行的方案,用以在光晶格中实现拓扑平坦能带以及分数量子反常霍尔态。最近有一篇工作系统研究了光晶格中朗道能级上的玻色型分数量子霍尔效应,其中明确地观测到了该效应的边缘激发谱。在本文,我们系统研究了无朗道能级的拓扑平坦能带(TFB)上的玻色型分数量子反常霍尔效应的边缘激发行为。我们的实空间严格对角化(ED)计算发现了碟形几何结构上的边缘激发谱的直接证据,而同行基于轨道纠缠谱的研究也给出了间接证据。我们考虑了硬核玻色子填充到Haldane碟形模型和kagom\'{e} 格子碟形模型中,发现了一系列特征边缘激发谱,符合手征Luttinger液体理论。通过在蝶形结构中心插入磁通并调节,我们进一步检验了这些激发态的可压缩性,发现确实是载流的手征边缘态。