\chapter{绪论}
\label{introduction}

\section{背景介绍}
在凝聚态物理领域,探索与分类各种物质的状态是一项非常引人注目的重要主题。整数和分数量子霍尔效应(quantum Hall effect, QHE)更是其中的研究焦点。众所周知,在二维自由电子气中,施加一个均匀的外磁场会产生高度简并的朗道能级,电子整数填充朗道能级时会产生整数量子霍尔效应~\cite{Klitzing1980} 。1982年发现的分数量子霍尔效应~\cite{Tsui1982},发现量子霍尔平台也能出现在分数填充的情形。这是自由电子气理论无法解释的。物理学者很快意识到,分数量子霍尔效应是一种强关联效应,与整数量子霍尔效应产生的机理截然不同。在平坦的朗道能级之上,电子的动能完全被压制,当相互作用的粒子分数填充在朗道能级上时,体系便有可能出现分数量子霍尔效应。迄今为止,由于只有在强磁场和极低温的二维电子气中观测到该效应,这一效应可能的现实应用受到了极大的限制。另一方面,在无外加均匀磁场的晶格体系中实现整数和分数量子霍尔效应引起了物理研究者极大的兴趣。
Haldane~\cite{Haldane1988}在1988年的工作首先在这一领域取得了突破。他认为量子霍尔效应的产生并不需要均匀磁场,满足时间反演对称破缺的体系就有可能具有非零的霍尔电导。他具体分析了二维蜂窝型结构的晶格模型,考虑在体系中加入交错磁通,同时保证整个原胞内的净磁通为零。计算每条能带的霍尔电导,发现了陈数为零的两条非平庸能带。Haldane模型实现了第一个晶格中拓扑绝缘体,这种时间反演对称破缺的拓扑绝缘体也称为陈绝缘体。既然晶格模型中能够实现整数量子霍尔效应,一个很自然的问题是:是否也存在分数陈绝缘体?由于分数化是一种强关联效应,要求相互作用的影响占支配作用,在高度色散的Haldane模型中并没有得到很好的结果。

新近提出的拓扑平坦能带~\cite{Tang2011,Sun2011,Neupert2011}满足了探索这一非朗道能级上的分数化现象的基本条件。拓扑平坦能带至少含有一支陈数~\cite{Thouless1982}非零的、平坦度较高的拓扑能带,这可以近似看作是连续朗道能级的晶格体系对应。最近在拓扑平带模型上开展的一些数值工作,系统地研究了其中的强关联效应,牢固地确立了在无朗道能级情形下的阿贝尔型~\cite{Sheng2011,Wang2011,Regnault2011}和非阿贝尔型分数量子霍尔效应\cite{Wang2012a,Bernevig2012,Wu2012},最稳定的拓扑相包括费米型的$\frac{1}{3}$量子霍尔效应,以及玻色型的$\frac{1}{2}$量子霍尔效应。它们具有多重准简并的基态组,与高能激发态之间有较大的体能隙,具有平滑的Berry曲率。对于陈数为1的情形,基态组共享单位陈数。除此之外,该体系的分数统计也充分地真实了分数量子霍尔态的存在。

于此同时,许多新的解析方法~\cite{Qi2011,Barkeshli2012a,Wu2013,Lee2013,Jian2013} \cite{Parameswaran2011,Goerbig2012,Murthy2011,Murthy2012} \cite{Lu2012,McGreevy2012,Zhang2013} \cite{Scaffidi2012,Wu2012a,Liu2013}
由不同小组相继提出,用来更好地理解拓扑平带模型中的分数化现象。最近这一领域找到了许多高平坦度的陈绝缘体模型,提出了在实验中可能实现的方案~\cite{Liu2013a,Cooper2013,Shi2013,Yannopapas2012,Zhang2013a,Grushin2012,Trescher2012,Chen2012,Weeks2012,Hu2011,Venderbos2011,Ghaemi2012,Wang2011a,Venderbos2012,Liu2012a,Yang2012,Xiao2011,Yao2013}。由于朗道能级的陈数总是1,而在拓扑平带模型中可以实现任意高的陈数,这些没有连续模型对应的高陈数模型,应当会存在不同于朗道能级上的分数化现象。最近在高陈数上的数值研究~\cite{Wang2012,Liu2012,Sterdyniak2013}已经得到了很好的验证,还有一些工作探索了其中的叠代分数量子霍尔态\cite{Liu2013b,Lauchli2013}。

%补充:实验方面。

%新发现的分数量子霍尔效应不同与传统朗道能级上的连续型分数量子霍尔效应,无需外加磁场,有较大的特征能隙,因此有可能在常温下存在。不需要朗道能级,不能用常规的Laughlin波函数描述。这些无外加磁场、无朗道能级的分数化现象,定义了一类新的分数拓扑相,也称为分数陈绝缘体,其中的分数量子霍尔效应也称为分数量子反常霍尔效应。由于其可能的现实应用,该领域在近期引起了物理研究者极大地兴趣。

%一些新的研究手段,例如基于Wannier表象的模型波函数和赝势法、投影密度算符代数、部分子(parton)波函数构造、绝热连通路径等方法被快速发展起来,以进一步理解这些分数量子反常霍尔态。多个研究小组提出了其他的拓扑平带模型以及可能的材料实现方案。最近的数值工作又发现了高陈数(high Chern number)的拓扑平带上的分数量子反常霍尔效应,由于每一条朗道能级的陈数总是1,这些分数化效应并没有直接的朗道能级上的连续对应。另外还观测到了非阿贝尔型(Non-abelian)以及叠代(hierarchy)分数量子反常霍尔态。


\section{关于本文}
尽管此前的研究已经牢固地确定了分数量子反常霍尔态(FCI/FQAH)的各种拓扑性质,但是边缘激发的研究在原则上可以提供另一个揭示体态拓扑序的窗口~\cite{Wen1995}。考虑到将来可能会首先在光晶格冷原子体系实现分数量子反常霍尔态(FCI/FQAH),边缘激发也可以提供一个切实可行的实验探测手段。在实验方面,已经提出了一些非常可行的方案,用以在光晶格中实现拓扑平坦能带以及分数量子反常霍尔态。最近有一篇工作系统研究了光晶格中朗道能级上的玻色型分数量子霍尔效应,其中明确地观测到了该效应的边缘激发谱~\cite{Kjall2012}。在本文,我们系统研究了无朗道能级的拓扑平坦能带(TFB)上的玻色型分数量子反常霍尔效应的边缘激发行为。我们的实空间严格对角化(ED)计算发现了碟形几何结构上的边缘激发谱的直接证据,而同行基于轨道纠缠谱的研究也给出了间接证据。我们考虑了硬核玻色子填充到Haldane碟形模型和kagom\'{e} 格子碟形模型中,发现了一系列特征边缘激发谱,符合手征Luttinger液体理论~\cite{Wen1995}。通过在蝶形结构中心插入磁通并调节,我们进一步检验了这些激发态的可压缩性,发现确实是载流的手征边缘态。

本文主要选取了两个典型的拓扑平带模型,仔细研究它们在蝶形几何结构中填入硬核玻色子所产生的边缘激发效应。具体的结构安排如下:第二章具体引入了本文研究的两个拓扑平带模型,介绍模型的晶格体系、模型哈密顿量、陈数的计算方法,分别画出了两个模型的体能带以及边缘态结构。第三章是本文的主体部分,首先展示了我们在严格对角化计算过程中得到的数值结果,包括蝶形几何结构中填入硬核玻色子的边缘激发谱,以及在系统中加入可调制的中心磁通,观测边缘态的磁通演化谱。本章的第二部分,系统地分析了之前得到的数值结果,包括边缘简并序列,基态总角动量,以及能谱的演化周期,可以发现,即使在非常小的系统中,也可以得到与理论符合得非常好的结果。这里使用到的主要理论手段包括文小刚的手征Luttinger液体理论,以及Haldane的广义Pauli原理。第四章是总结与展望,最近该领域发展得十分迅速,有许多值得探索的问题,这里做一个将来的规划。