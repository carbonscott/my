\chapter{拓扑平带模型}
\label{sec:models}
% 如何得到平带;如何计算陈数
在这一章节,我们呈现了两个典型的拓扑平坦能带模型,分别计算了各个能带的陈数,以及模型所能得到的能带平坦度。这两个模型是本文用来研究分数陈绝缘体的单体哈密顿量。


\section{陈数的计算}
零温下的霍尔电导,可以用Kubo公式~\cite{Thouless1982}进行计算:
\begin{eqnarray}
    \sigma_{\rm
H}({E})&=&\frac{{i}e^{2}\hbar}{{A}_0}\sum_{{\cal
E}_{{m\mathbf{k}}}<E}\sum_{{\cal E}_{{n\mathbf{k}}}>E}\nonumber\\%\nonumber
&&\frac{\langle{m\mathbf{k}}|{v}_{{x}}|{n\mathbf{k}}\rangle\langle{n\mathbf{k}}|{v}_{{y}}|{m\mathbf{k}}\rangle
-\langle{m\mathbf{k}}|{v}_{{y}}|{n\mathbf{k}}\rangle\langle{n\mathbf{k}}|{v}_{{x}}|{m\mathbf{k}}\rangle}{({\cal
E}_{{m\mathbf{k}}}-{\cal E}_{{n\mathbf{k}}})^{2}}
\label{eq:kubo1}
\end{eqnarray}
其中$A_{0}$是系统所占的面积,$E$为费米能级,${\cal E}_{m\mathbf{k}}$和$|m\mathbf{k}\rangle$分别为体系哈密顿量的本征值与本征函数。对$\mathbf{k}$的求和遍及这个第一布里渊区。

由于上式\ref{eq:kubo1}涉及到速度算符$v_{\tau}$对波函数的求导,应当给波函数选取一个平滑的相位规范。这一规范的要求限制了\ref{eq:kubo1}式在实际数值计算中的应用。下面介绍一种规范不变的形式,引入$v_{\tau}(\mathbf{k})=\frac{1}{\hbar}\frac{\partial{H_{\mathbf{k}}}}{\partial{k_{\tau}}}$,我们可以把Kubo公式重新写成以下规范不变的形式:
\begin{eqnarray}
    \sigma_{\rm
H}({E})&=&\frac{{i}e^{2}}{{A}_0\hbar}\sum_{{\cal
E}_{{m\mathbf{k}}}<E}\sum_{{\cal E}_{{n\mathbf{k}}}>E}\nonumber\\%\nonumber
&&\frac{\langle{m\mathbf{k}}|\frac{\partial{H}}{\partial{k_{x}}}|{n\mathbf{k}}\rangle\langle{n\mathbf{k}}|\frac{\partial{H}}{\partial{k_{y}}}|{m\mathbf{k}}\rangle
-\langle{m\mathbf{k}}|\frac{\partial{H}}{\partial{k_{y}}}|{n\mathbf{k}}\rangle\langle{n\mathbf{k}}|\frac{\partial{H}}{\partial{k_{x}}}|{m\mathbf{k}}\rangle}{({\cal
E}_{{m\mathbf{k}}}-{\cal E}_{{n\mathbf{k}}})^{2}}
\label{eq:kubo2}
\end{eqnarray}

Thouless等人~\cite{Thouless1982}在1982年提出,量子霍尔效应之所以存在霍尔平台,是由于实际的观测量与拓扑不变量相关联,也就是霍尔电导与陈数相关联。可以证明,霍尔电导可以写成如下形式:
\begin{equation*}
    \sigma_{\rm
H}({E})=\sum_{{\cal E}_{{m}}<E}\sigma_{\rm
H}^{(m)}=e^{2}/{h}\sum_{{\cal E}_{{m}}<E}C_{m}
\end{equation*}
其中 $\sigma_{\rm
H}^{(m)}$ 和 $C_{m}$分别为第$m$个完全填充能带的霍尔电导和陈数。


\section{扩展Haldane模型}
Haldane模型定义在二维蜂窝型晶格中,也就是最近发现的石墨烯的晶格结构,对应的布拉维晶格为六角晶格,基矢为$\textbf{a}_1,\textbf{a}_2$,如图~\ref{fig:HClattice}所示。为了说明量子霍尔效应的存在并不依赖于朗道能级,Haldane在1988年提出了一个紧束缚模型,考虑在六角晶格中引入交错磁通,即在次近邻的跃迁中引入相位,同时保证整个体系中的净磁通为零。通过简单的计算,Haldane得到了两条陈数非零的非平庸能带,得到了无朗道能级的整数粒子霍尔效应,现在也称为量子反常霍尔效应(Quantum anomoulous Hall effect),或陈绝缘体(Chern insulator)。

\begin{figure}
 \centering
 \includegraphics[width=0.5\textwidth]{figure/HC_lattice}
 \caption{蜂窝型晶格}
 \label{fig:HClattice}
\end{figure}

为了能够在Haldane模型中得到平坦度较高的拓扑平带,这里进一步引入了次次近邻格点间的跃迁,我们把它称为扩展Haldane模型。具体的搜索方法是,在一个较大的参数空间内,通过遍历其中的组合参数,计算相应的平坦度,其中最大的认为是拓扑平带。对于二能级系统,平坦度定义为体能带的能隙与当前能带的比值。哈密顿量为:
\begin{eqnarray}
H_{\rm HC}= &-&t^{\prime}\sum_{\langle\langle\mathbf{r}\mathbf{r}^{
\prime}\rangle\rangle}
\left[c^{\dagger}_{\mathbf{r}^{ \prime}}c_{\mathbf{r}}\exp\left(i\phi_{\mathbf{r}^{ \prime}\mathbf{r}}\right)+\textrm{H.c.}\right]\\
&-&t\sum_{\langle\mathbf{r}\mathbf{r}^{ \prime}\rangle}
\left[c^{\dagger}_{\mathbf{r}^{\prime}}c_{\mathbf{r}}+\textrm{H.c.}\right]
-t^{\prime\prime}\sum_{\langle\langle\langle\mathbf{r}\mathbf{r}^{
\prime}\rangle\rangle\rangle}
\left[c^{\dagger}_{\mathbf{r}^{\prime}}c_{\mathbf{r}}+\textrm{H.c.}\right]\nonumber
\label{eq:hc}
\end{eqnarray}
其中$c^{\dagger}_{\mathbf{r}}$为格点$\mathbf{r}$处的费米子产生算符,  $\langle\dots\rangle$,$\langle\langle\dots\rangle\rangle$ 和 $\langle\langle\langle\dots\rangle\rangle\rangle$ 分别表示最近邻、次近邻以及第三近邻的格点对。
选取\cite{}文献中的参数,可以得到一个平坦度为 的拓扑平带。根据\ref{eq:kubo2}可以计算出两条能带的陈数分别为$\pm1$

\begin{figure}[h]
  \centering
  \includegraphics[width=\textwidth]{figure/HC_spectrum}
\caption{扩展Haldane模型}
\label{fig:hc_spectrum}
\end{figure}
%\begin{figure}[ht]
%  \begin{minipage}[c]{0.5\textwidth}
%  \includegraphics[width=\textwidth]{figure/HC_bulk}
%  \hspace{5in}
%  \end{minipage}
%  \begin{minipage}[c]{0.5\textwidth}
%  \includegraphics[width=\textwidth]{figure/HC_edge}
%  \end{minipage}
%\caption{扩展Haldane模型}
%\label{fig:TFB_HC}
%\end{figure}


\section{Kagome格子模型}
Kagome晶格由 组成,每个元胞中含有三套子个,晶格示意图以及基矢如图~\ref{fig:KGlattice}所示。


\begin{figure}
 \centering
 \includegraphics[width=0.5\textwidth]{figure/KG_lattice}
 \caption{kagome型晶格}
 \label{fig:KGlattice}
\end{figure}

哈密顿量为:
\begin{eqnarray}
H_{\rm KG}= &-&t\sum_{\langle\mathbf{r}\mathbf{r}^{ \prime}\rangle}
\left[c^{\dagger}_{\mathbf{r}^{ \prime}}c_{\mathbf{r}}\exp\left(i\phi_{\mathbf{r}^{ \prime}\mathbf{r}}\right)+\textrm{H.c.}\right]\nonumber\\
&-&t^{\prime}\sum_{\langle\langle\mathbf{r}\mathbf{r}^{\prime}\rangle\rangle}
\left[c^{\dagger}_{\mathbf{r}^{\prime}}c_{\mathbf{r}}+\textrm{H.c.}\right]%\nonumber\\
\label{eq:kg}
\end{eqnarray}

选取\cite{}文献中的参数,可以得到一个平坦度为 的拓扑平带。根据\ref{eq:kubo2}可以计算出两条能带的陈数分别为$\pm1$.
\begin{figure}[h]
  \centering
  \includegraphics[width=\textwidth]{figure/HC_spectrum}
\caption{扩展Haldane模型}
\label{fig:hc_spectrum}
\end{figure}

%\begin{figure}[ht]
%  \begin{minipage}[c]{0.5\textwidth}
%  \includegraphics[width=\textwidth]{figure/KG_bulk}
%  \hspace{5in}
%  \end{minipage}
%  \begin{minipage}[c]{0.5\textwidth}
%  \includegraphics[width=\textwidth]{figure/KG_edge}
%  \end{minipage}
%\caption{Kagome格子模型}
%\label{fig:TFB_KG}
%\end{figure}

为了更好的显示拓扑能带与普通绝缘体的区别,这里也做了边缘态的计算。