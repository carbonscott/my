\chapter{拓扑平带模型}
\label{models}
% 如何得到平带;如何计算陈数
在这一章节,我们呈现了两个典型的拓扑平坦能带模型,分别计算了各个能带的陈数,以及模型所能得到的能带平坦度。这两个模型是本文用来研究分数陈绝缘体的单体哈密顿量。

\section{扩展Haldane模型}



\begin{figure}[ht]
  \begin{minipage}[c]{0.5\textwidth}
  \includegraphics[width=\textwidth]{figure/HC_bulk}
  \hspace{5in}
  \end{minipage}
  \begin{minipage}[c]{0.5\textwidth}
  \includegraphics[width=\textwidth]{figure/HC_edge}
  \end{minipage}
\caption{扩展Haldane模型}
\label{fig:TFB_HC}
\end{figure}


\section{Kagome格子模型}

\begin{figure}[ht]
  \begin{minipage}[c]{0.5\textwidth}
  \includegraphics[width=\textwidth]{figure/KG_bulk}
  \hspace{5in}
  \end{minipage}
  \begin{minipage}[c]{0.5\textwidth}
  \includegraphics[width=\textwidth]{figure/KG_edge}
  \end{minipage}
\caption{Kagome格子模型}
\label{fig:TFB_KG}
\end{figure}