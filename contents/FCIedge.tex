\chapter{分数陈绝缘体中的边缘激发}
\label{sec:FCIedge}

本章的主要内容是研究陈绝缘体中的强关联效应。这里以上一章中提到的两个拓扑平带模型为基础,考虑在能带中填入短程相互作用的硬核玻色子,研究蝶形体系的边缘激发效应。

\section{数值研究}
\subsection{边缘激发谱}
之前关于拓扑平带的计算是在环面结构(torus)的体系中开展的,由此得到了平坦度较大的体能带。Haldane模型中,体系具有两套子格,得到的两条能带中能态的数目就等于系统原胞的个数,记为$N$,定义填充率为$\upsilon=N_{particle}/N$。类比于电子在最低朗道能级(LLL)中$\upsilon=\frac{1}{2}$填充的分数量子霍尔效应(对于玻色子情形,由于波函数对称性的要求,填充率为$\upsilon=\frac{1}{2}$),硬核玻色子在拓扑平带中的填充率应为$\upsilon=\frac{1}{2}$。我们组之前的工作已经在数值上牢固地确立了扩展Haldane模型在环面结构中的玻色型$\frac{1}{2}$分数量子反常霍尔效应~\cite{Wang2011}。本文的主要工作在于研究相对研究较少的边缘激发效应。

% 蝶形几何结构
\begin{figure}[!tb]
  \centering
  \begin{minipage}[c]{0.45\textwidth}
  \includegraphics[width=\textwidth]{figure/HC96_disk.eps}
\end{minipage}
\hfill
  \begin{minipage}[c]{0.45\textwidth}
  \includegraphics[width=\textwidth]{figure/KG72_disk.eps}
\end{minipage}
  \caption{(a)六角蜂窝型的蝶形几何结构,(b)kagom\'{e}蝶形几何结构。两个体系的蝶形几何结构都满足$C_6$旋转对称性。图中用虚线表示的圆形以及标注的数字代表不同尺寸的结构。}
\label{fig:disk}
\end{figure}

更具体的说,我们选取了两个典型的拓扑平坦能带模型,分别选取具有六重旋转对称性的蝶形几何结构,因此每个能态都能用角动量量量子数$L$来标记。为了更好的观测到边缘激发谱,在该有限尺寸系统中,加入一个额外的势阱来约束FCI/FQAH液滴,从而在FCI/FQAH液滴的边缘可以有边缘激发模式围绕蝶形结构传播。不是一般性,我们采用了典型的中心谐振势~\cite{Kjall2012},形式为$V = V_{\rm trap}\sum_{\mathbf{r}} |\mathbf{r}|^2 n_{\mathbf{r}}$,其中$V_{\rm trap}$ 为势场强度(以最近邻跃迁参数$t$作为能量单位), $|{\mathbf{r}}|$ 为蝶形几何中心到格点的半径(以半晶格常数$a/2$作为长度单位),  $n_{\mathbf{r}}$ 为玻色子数目算符。在具体的数值计算过程中,我们发现,在相当大的谐振势范围内,改变束缚势的强度并不会影响边缘激发序列,这也显示了量子霍尔态的稳定性。因此在下面的结果中,一般把中心谐振势的强度固定为$V_{\rm trap}=0.005$。
利用严格对角化~的方法研究小尺寸的蝶形几何结构,我们发现了非常明显的边缘序列。

我们研究的第一个模型是六角蜂窝格子上的扩展Haldane模型,在其中填入硬核玻色子~\cite{Wang2011},其哈密顿量为:
\begin{eqnarray}
H_{\rm HC}= &-&t^{\prime}\sum_{\langle\langle\mathbf{r}\mathbf{r}^{
\prime}\rangle\rangle}
\left[b^{\dagger}_{\mathbf{r}^{ \prime}}b_{\mathbf{r}}\exp\left(i\phi_{\mathbf{r}^{ \prime}\mathbf{r}}\right)+\textrm{H.c.}\right]\\
&-&t\sum_{\langle\mathbf{r}\mathbf{r}^{ \prime}\rangle}
\left[b^{\dagger}_{\mathbf{r}^{\prime}}b_{\mathbf{r}}+\textrm{H.c.}\right]
-t^{\prime\prime}\sum_{\langle\langle\langle\mathbf{r}\mathbf{r}^{
\prime}\rangle\rangle\rangle}
\left[b^{\dagger}_{\mathbf{r}^{\prime}}b_{\mathbf{r}}+\textrm{H.c.}\right]\nonumber
\label{eq:HC}
\end{eqnarray}
其中$b^{\dagger}_{\mathbf{r}}$为格点$\mathbf{r}$处的硬核玻色子产生算符,  $\langle\dots\rangle$,$\langle\langle\dots\rangle\rangle$ 和 $\langle\langle\langle\dots\rangle\rangle\rangle$ 分别表示最近邻、次近邻以及第三近邻的格点对。选取文献~\cite{Wang2011}中的跃迁参数,我们可以得到一个平坦度为50的拓扑平带,即$t=1$,$t^{\prime}=0.60$, $t^{\prime\prime}=-0.58$ 和 $\phi=0.4\pi$。

在较大的谐振势强度范围内,我们都在蜂窝格子上的扩展Haldane模型中观测到了明显的边缘激发谱。这里选取了多个不同的小尺寸蝶形结构,分别包括 $N_s = 24, 54, 96$个格点,参见体系示意图\ref{fig:disk}[a]。在这些小体系中,通过填充硬核玻色子,我们都得到了类似的简并序列。一个典型的严格对角化结果如图\ref{fig:HC_edge}所示,这里展现了$N_s=96$个格点的情形,保持约束势强度为$V_{\rm trap}=0.005$。四个子图显示了不同的激发序列,分别对应玻色子数为$N_b = 2,3,4,5$的情形。由于体系具有$C_6$旋转对称性,各个能态都可以用一个角动量量子数$L$来标记,其中$L = 0, 1, 2, 3,..., \mod 6$。图中标在各个区间能态上面的数字表示准简并序列,下面将提到,这一序列与手征Luttinger液体理论所描述的结果是一致的。

% HC分数陈绝缘体的边缘激发谱
\begin{figure}[!htb]
  \centering
  \includegraphics[width=0.6\textwidth]{{HC96_edge_vtrap_0.0050}.eps}
  \caption{96格点的扩展Haldane模型中的边缘激发谱,束缚势强度为$V_{\rm trap}=0.005$。图中的数字标明了当前角动量分区的能态准简并度,这一数字也可以手征Chiral液体理论中得到。图中各个子图填充的硬核玻色子数目不同,分别对应玻色子数为$N_b = 2,3,4,5$的情形。}
\label{fig:HC_edge}
\end{figure}


我们考虑的第二个模型是kagom\'{e} 格子模型~\cite{Tang2011,Sun2011,Neupert2011,Liu2012a},在其中填入硬核玻色子,其哈密顿量为:
\begin{eqnarray}
H_{\rm KG}= &-&t\sum_{\langle\mathbf{r}\mathbf{r}^{ \prime}\rangle}
\left[b^{\dagger}_{\mathbf{r}^{ \prime}}b_{\mathbf{r}}\exp\left(i\phi_{\mathbf{r}^{ \prime}\mathbf{r}}\right)+\textrm{H.c.}\right]\nonumber\\
&-&t^{\prime}\sum_{\langle\langle\mathbf{r}\mathbf{r}^{\prime}\rangle\rangle}
\left[b^{\dagger}_{\mathbf{r}^{\prime}}b_{\mathbf{r}}+\textrm{H.c.}\right]%\nonumber\\
\label{eq:KG}
\end{eqnarray}
选取文献\cite{Liu2012a}中的跃迁参数,我们可以得到一个平坦度为20的拓扑平带,即$t=1$, $t^{\prime}=-0.19$, $\phi=0.22\pi$。

% KG分数陈绝缘体的边缘激发谱
\begin{figure}[!htb]
  \centering
  \includegraphics[width=0.6\textwidth]{{KG72_edge_vtrap_0.0050}.eps}
  \caption{72格点的扩展Haldane模型中的边缘激发谱,束缚势强度为$V_{\rm trap}=0.005$。图中的数字标明了当前角动量分区的能态准简并度,这一数字也可以手征Chiral液体理论中得到。图中各个子图填充的硬核玻色子数目不同,分别对应玻色子数为$N_b = 3,4,5,6$的情形。}
\label{fig:KG_edge}
\end{figure}

利用同样的手段,可以得到kagom\'{e} 格子模型的边缘激发。图\ref{fig:KG_edge}展示了$N_s=72$个格点的激发谱结果,四个子图分别对应玻色子数目为$N_b = 3,4,5,6$的情形。可以发现,尽管kagom\'{e} 格子体系的格点数要比上一个模型的少,这里得到了比$N_s=96$的扩展Haldane模型更为清晰的激发谱,同时能够研究的玻色子数目更多(在严格对角化的计算中)。除去两个模型的巨大差异,通过比较上述两幅典型的激发谱,还可以发现以下的明确的证据:
\begin{itemize}
\item 对于不同的玻色子填充,体系的基态位于不同的角动量分区。对于填充数为$N_b=2,3,4,5,6$的体系,相应的基态分别位于角动量分区$L=0,3,2,3,0$。
\item 边缘激发谱存在一个清晰的准简并序列,这一具体的序列数取决于体系填充的玻色子数目。
\end{itemize}
以上的观察结果适用于讨论这里所有的结果,包括不同的晶格模型,不同的晶格尺寸,以及较大范围的谐振势。鉴于数值计算研究的体系大小非常有限,呈现的激发谱中表现出了有限尺寸效应,得到的能级并非严格简并。同样的,鉴于严格对角化只能精确的计算基态以及较低的几个激发态,对于高能级的激发态,计算结果会产生较大的误差;小尺寸也会对此加以影响。但是这并不妨碍我们对此做出定性的判断。事实上,用精确的Lapack对角化处理较小的希尔伯特空间时,我们得到了非常好的数据结果。比如在$N_s=54$的扩展Haldane模型中填充3个玻色子,用上角动量对称性后矩阵维数为$4\times10^3$的量级,可以用精确对角化进行处理,得到的数据图如下。经过比较发现,这一序列谱与理论预言符合的非常好,完全一致的序列达到了16个之多!计算这一简并序列的小程序可参见附录,用Python语言编写。

\begin{figure}[!htb]
  \centering
  \includegraphics[width=0.6\textwidth]{lapack_edge.pdf}
  \caption{3个硬核玻色子填充54格点的扩展Haldane模型时得到的边缘激发谱。该结果用Lapack完全对角化得到。}
\label{fig:lapack_edge}
\end{figure}

%此前的数值工作已经牢固地确立了环面结构上分数填充拓扑平带的玻色型$\frac{1}{2}$分数量子反常霍尔效应,这些证据包括:准简并的多重基态,较大的体特征能隙图,特征基态动量,以及各个基态的分数量子化的陈数。两个模型的蝶形几何结构都具有六重旋转对称性,如图\ref{fig:disk}。为了更好地观测到分数量子霍尔效应的边缘激发谱,我们分别在两个蝶形结构中加入了中心约束势场,使得边缘模式能够在外围传播。这里我们采用了典型的中心谐振势,形式为$V = V_{\rm trap}\sum_{\mathbf{r}} |\mathbf{r}|^2 n_{\mathbf{r}}$,其中$V_{\rm trap}$ 为势场强度(以最近邻跃迁参数$t$作为能量单位), $|{\mathbf{r}}|$ 为蝶形几何中心到格点的半径(以半晶格常数$a/2$作为长度单位),  $n_{\mathbf{r}}$ 为玻色子数目算符。




\subsection{中心磁通}
\label{sec:flux}
接下来,我们进一步在蝶形几何中心点加入大小可变的磁通,用以观测边缘态的可压缩性。这里考虑$N_s=42$格点的kagom\'{e}格子模型,并且保持谐振势的强度为$V_{\rm trap}=0.005$。对于$N_b=3$个玻色子的体系,$L= 0, 3$角动量区间的低能边缘态(或 $L= 1, 4$ 和 $L= 2, 5$ 区间的低能态)相互演化,如图\ref{fig:flux}(a)所示:当中心磁通$\Phi$的大小改变$2\pi$时,角动量量子数$L= 0$的低能态(用红色短线表示)与$L= 3$的低能态(用绿色短线表示)相互演化,即$E_n(L+3,\Phi+2\pi)=E_n(L,\Phi)$,其中$L,L+3=0,1,2,\dots\mod 6$。可以看出,改变磁通大小时,低能边缘态无能隙地演化到更高的能级,显示出了边缘态的载流特性(边缘态的可压缩性)。

% 多体态磁通演化能谱
\begin{figure}[!htb]
  \centering
  \includegraphics[width=0.8\textwidth]{{KG42_vtrap0.005_flux}.eps}
  \caption{在42格点的kagom\'{e}蝶形结构中,低能边缘态随磁通改变的能谱演化图,约束势为$V_{\rm trap}=0.005$。(a)体系的填充玻色子数$N_b = 3$时,角动量为$L= 0, 3$分区的能态(分别用红色与绿色短线表示)相互演化。(b)体系的填充玻色子数$N_b = 4$时,角动量为$L= 0, 2,4$分区的能态(分别用红色,绿色与蓝色短线表示)相互演化。}
\label{fig:flux}
\end{figure}

对于$N_b=4$个玻色子的体系,角动量为$L= 0, 2, 4$(或$L= 1, 3, 5$)的能级如图\ref{flux}(b)相互演化:当中心磁通的大小改变$2\pi$,$4\pi$时,$L= 0$的能级(红色短线表示),$L= 2$的能级(绿色短线表示)与$L= 4$的能级(蓝色短线表示)相互循环演化,即$E_n(L+2,\Phi+2\pi)=E_n(L+4,\Phi+4\pi)=E_n(L,\Phi)$,其中$L,L+2,L+4=0,1,2,\dots\mod 6$。经过三个磁通周期时,各个区间的能态演化回自身,即$E_n(L,\Phi+6\pi)=E_n(L,\Phi)$。对于这里考虑的具有$C_6$旋转对称性的体系,当填充的玻色子数目满足$N_b/6=p/q$($p$和$q$为互质整数)时,存在一个通用的能谱周期演化关系:$E_n(L,\Phi+2q\pi)=E_n(L,\Phi)$。在下一节内容中可以看到,这一多体态的演化规律可以定性地从单粒子轨道的图像来理解。


\section{理论解释}

\subsection{手征Luttinger液体理论}
朗道能级上的分数量子霍尔效应,它的基态可以用无旋、不可压缩的液体描述,不存在低能的体激发态。唯一可能存在的中性低能激发态是边缘态,这些表面波可以认为是对基态霍尔小液滴的变形或扩展。文小刚首先意识到了这一点,并且将这一经典描述的过程量子化,得到了描述这些低能激发态的有效哈密顿量。

%在较大的谐振势强度范围内,我们都在蜂窝格子上的扩展Haldane模型中观测到了明显的边缘激发谱。这里选取了多个不同的小尺寸蝶形结构,分别包括 $N_s = 24, 54, 96$个格点。图\ref{fig:HC_edge},从中可以清新地观测到不同玻色子填充数所对应的边缘激发能谱。由于体系具有$C_6$旋转对称性,各个能态都可以用一个角动量量子数$L$来标记,其中$L = 0, 1, 2, 3,..., \mod 6$。可以期待,在有限尺寸的体系中,数值结果得到的不同玻色子数目对应的边缘简并序列,应当与手征Luttinger液体理论相符合。

%\begin{figure}[!htb]
%  \centering
%  \includegraphics[width=0.7\textwidth]{{HC96_edge_vtrap_0.0050}.eps}
%  \caption{(color online). Edge excitations on a honeycomb disk with $N_s=96$ sites and trap potential $V_{\rm trap}=0.005$. Numbers labelled upon low energy levels in each sector show the quasi-degeneracy of low edge excitations, which is also the sequence derived from the chiral Luttinger liquid theory.}
%\label{fig:HC_edge}
%\end{figure}

根据文小刚的流体力学方法~\cite{Wen1995},对于填充数为$\frac{1}{2}$的玻色型分数量子霍尔效应,其低能边缘激发态由Kac-Moody代数描述,形成用以下有效哈密顿量描述的手征Luttinger液体:
\begin{equation}\label{eq:luttinger}
   H_{1/m}=2\pi \frac{v}{\nu}\displaystyle\sum_{k>0}\rho_{-k}\rho_k=v\sum_{k>0} k a^{\dagger}_k a_k,
\end{equation}
其中$k$为边界上的角动量,$\rho_k$为傅里叶变换后的一维密度算符,满足$[\rho_k,\rho_{k'}]=\frac{\nu}{2\pi}k\delta_{k+k'}$。 $a^{\dagger}_k$和$a_k$是手征玻色子(声子)的产生/堙没算符。基于该理论,手征边缘玻色子占据在角动量为$k=1, 2, 3, 4, 5, 6, ...$ (以 $2\pi/M$ 为单位,其中$M$为边界长度)的单粒子轨道中。我们将其中的占据分布标记为\{$n_1,n_2,n_3,n_4,n_5,n_6$, ...\},把相对(基态的)总角动量标记为$\Delta L = L - L_{\rm GS}$。对于特定的相对总角动量$\Delta L = \sum_{k} kn_{k}$,我们可以通过数出允许的占据构型的个数,来计算相应的简并度。参考表~\ref{tab:luttinger}中的前两列,这里列出了$\Delta L = 0,1,\cdots,6$的所有可能的状态数,通过相同的方法可以计算出任意$\Delta L$下的简并度。在热力学极限下,对于具有足够多的玻色子数目的体系,这一序列应为:1,1,2,3,5,7,11,15,22,...。

在实际的严格对角化计算中,所能研究的系统通常很小,相应的填充粒子数也非常有限。以上的结论仅适用于热力学极限,并不能直接拿来与数值结果对比。对于玻色子数目为$N_b$的情形,可以通过统计$\sum_k n_k\le N_b$的单模态数来确定相应的边缘简并序列。表格~\ref{tab:luttinger}的第三分别列出了不同玻色子数填充时的小角动量量子数的简并度。可以看到,对于$N_b=3$个玻色子的情形,除去表中$\sum_k n_k>N_b=3$的构型,期望的简并度序列应为1,1,2,3,4,5,7,...,这与我们从严格对角化中得到的数值结果是一致的。同理,对于$N_b=4$个玻色子的情形,除去表中$\sum_k n_k>N_b=4$的构型,期望的简并度序列应为1,1,2,3,5,6,...,这也是与数值结果的证据相符合的。表格~\ref{tab:luttinger},绝大多数在我们的严格对角化结果中观察到。如果考虑小体系的完全对角化,这一符合的序列数更是达到了16个之多。

以上的讨论,我们只是讨论分数量子霍尔效应的普遍理论,并没有依赖于特定的那个模型体系,甚至没有针对分数陈绝缘体的具体情形。从本文得到的边缘激发图中看到,这一谱序列非常的稳定,即使对于非常小体系的晶格,截然不同的体系,这一边缘序列也能够很好的展现出来。这足以看到分数量子霍尔效应的稳定性,也正是拓扑不变性的体现。

\begin{table}[h]
\centering
\begin{tabular}{c|l|l c c c c c}
 \hline
 \hline
$\Delta L$ & \{$n_1,n_2,n_3,n_4,n_5,n_6$,...\}        &  &2b &3b &4b &5b &6b  \\
 \hline
 \hline
     0 & \{0,0,0,0,0,0,...\}                          &$d=$  &1 &1 &1 &1  &1  \\

 \hline
     1 & \{1,0,0,0,0,0,...\}                          &$d=$  &1 &1 &1 &1  &1  \\
 \hline
     2 & \{0,1,0,0,0,0,...\},   \{2,0,0,0,0,0,...\}  &$d=$  &2 &2 &2 &2  &2  \\
 \hline
     3 & \{0,0,1,0,0,0,...\},   \{1,1,0,0,0,0,...\}  &$d=$  &2 &3 &3 &3  &3  \\
       & \{3,0,0,0,0,0,...\}                                                 \\
 \hline
     4 & \{0,0,0,1,0,0,...\},   \{1,0,1,0,0,0,...\}  &$d=$  &3 &4 &5 &5  &5  \\
       & \{0,2,0,0,0,0,...\},   \{2,1,0,0,0,0,...\}                         \\
       & \{4,0,0,0,0,0,...\}                                                 \\
 \hline
     5 & \{0,0,0,0,1,0,...\},   \{1,0,0,1,0,0,...\}  &$d=$  &3 &5 &6 &7  &7  \\
       & \{0,1,1,0,0,0,...\},   \{2,0,1,0,0,0,...\}                         \\
       & \{1,2,0,0,0,0,...\},   \{3,1,0,0,0,0,...\}                         \\
       & \{5,0,0,0,0,0,...\},                                                \\
 \hline
     6 & \{0,0,0,0,0,1,...\},   \{1,0,0,0,1,0,...\}  &$d=$  &4 &7 &\underline{9} &\underline{10} &\underline{11} \\
       & \{0,1,0,1,0,0,...\},   \{2,0,0,1,0,0,...\}                         \\
       & \{0,0,2,0,0,0,...\},   \{1,1,1,0,0,0,...\}                         \\
       & \{3,0,1,0,0,0,...\},   \{0,3,0,0,0,0,...\}                         \\
       & \{2,2,0,0,0,0,...\},   \{4,1,0,0,0,0,...\}                         \\
       & \{6,0,0,0,0,0,...\}                                                 \\
 \hline
 \hline
\end{tabular}
\caption{手征边缘玻色子在$k=1, 2, 3, 4, 5, 6,$ ...轨道中的占据分布:\{$n_1,n_2,n_3,n_4,n_5,n_6$, ...\}。$d$代表某所在角动量分区中的能级简并度。$\Delta L = L - L_{\rm GS} = \sum_{k} kn_{k}$,表示偏移基态角动量的中角动量大小。表格右栏表示在特定的粒子数条件下,边缘激发态的简并度,这里的大多数序列都在实际的计算过程中得到(除了下划线标注的以外,这是由严格对角化的计算性质决定的)}
\label{tab:luttinger}
\end{table}

%下面我们研究kagom\'{e}格子模型。同样地,我们选取了几个不同的尺寸:$N_s =12, 30, 42, 72$,在其中填充硬核玻色子。图显示了$N_s=72$格点的体系相应的边缘谱,这一能谱图甚至得到了比$N_s=96$格点的蜂窝型蝶形结构更好的结果。可以发现,即使在如此小的系统中,边缘激发谱并不依赖具体的晶格结构。对于填充$N_b=6$个玻色子的体系,去除掉$\sum_k n_k > N_b=6$的构型,期望得到的序列是1,1,2,3,5,7,...,正如图\ref{fig:KG_edge}(d)所示。大量的数值结果表明,体系的几何尺寸越大,与理论符合的角动量简并区间越多,这真是有限尺寸效应的表现。

%\begin{figure}[!htb]
%  \centering
%  \includegraphics[width=0.7\textwidth]{{KG72_edge_vtrap_0.0050}.eps}
%  \caption{(color online). Edge excitations on a kagom\'{e} disk with $N_s=72$ sites and trap potential $V_{\rm trap}=0.005$. Numbers labelled upon low energy levels in each sector show the quasi-degeneracy of low edge excitations, which is also the sequence derived from the chiral Luttinger liquid theory.}
%\label{fig:KG_edge}
%\end{figure}

在这一小结中,我们通过Luttinger液体理论分析了霍尔小液滴的普遍激发行为,发现这一激发序列与分数陈绝缘体的边缘激发行为符合得非常好。除此之外,我们也注意到,当体系的粒子数改变时,相应的基态总角动量也发生了改变。在下面一小结中,我们将借助广义不相容原理来系统地理解这些数值结果。对于$1/2$量子霍尔效应,广义不相容原理要求填充在连续两个单粒子轨道中的粒子数目不超过1,这一简单的原理使得我们能够更好地理解计算结果。


\subsection{广义Pauli原理}
在前文,我们呈现了在有限蝶形结构中,填充不同的硬核玻色子所得到的边缘能谱图。可以发现,对于不同的粒子填充数,体系的基态位于不同的角动量区间,具有的简并序列也不同。如果在体系引入中心磁通,不同数目的玻色子还会导致不同的能态演化周期,具体的关系式为:$E_n(L,\Phi+2q\pi)=E_n(L,\Phi)$。这些物理现象都与填充的玻色子数息息相关,事实上,我们可以借助广义Pauli原理~\cite{Regnault2011}\cite{Bernevig2008,Bergholtz2006,Seidel2006}
来理解这些看似奇怪的数值结果。


\subsubsection{单粒子轨道}
为了更好的理解这些多体态的性质,我们先来看看相应的单粒子轨道,期待从单粒子轨道的特性解释多粒子态。这里选取72格点的kagom\'{e} 蝶形体系作为实例,几何结构参见图~\ref{fig:disk}。在其中填充单个玻色子,可以得到相应的单粒子能谱图,如图~\ref{fig:KG72_1b_edge}所示。从右侧的能谱放大图可以看到,该体系的基态位于角动量量子数为$l = 5$的区间,其余低能激发态依次位于$l = 4 + i$的分区,其中$i$表示第$i$个单粒子能级。考虑到所研究体系的六重旋转对称性,角动量也可以写成$l = 4 + i \mod 6$。在这一单粒子体系加入中心磁通,发现一系列低能激发态互相演化,并且与其他高能态隔开。对于42格点的kagom\'{e}晶格,底下有18个能态相互演化(该体系对应多体态的能谱演化图~\ref{fig:flux}),见图~\ref{fig:KG42_1b_flux}。当磁通大小改变$2\pi$时,处于$l$分区的能态演化到了$l+1 \mod 6$分区。在接下来的几个小结中,我们将具体阐述如何从单粒子的行为来预测多体态的性质。

% 72格点的单粒子能谱图
\begin{figure}[!htb]
    \centering
    \includegraphics[width=0.8\textwidth]{{KG72_1b_edge_vtrap_0.0050}.eps}
    \caption{有限尺寸的kagom\'{e}蝶形结构系统的单粒子轨道谱,约束势为$V_{\rm trap}=0.005$。(a)72格点的单粒子能谱图。(b)图(a)中低能态的放大图。可以看到,基态角动量$l = 5$,其余低能激发态依次位于角动量分区$l = 4 + i$,$i$表示第$i$个能级。图中的红色小球代表有一个玻色子占据在相应的单粒子轨道中,这是多体态在广义Pauli原理的要求下的基态构型。}
\label{fig:KG72_1b_edge}
\end{figure}

根据广义Pauli原理,硬核玻色子将填充这一系列的单粒子轨道,并要求满足在连续的两个轨道中填充的玻色子不超过一个(对于$\frac{1}{2}$玻色型分数量子霍尔效应)。凭借这一简单的物理图像,以上提到的问题都能得到很好地阐明。

% 单粒子态磁通演化能谱
\begin{figure}[!htb]
    \centering
    \includegraphics[width=0.9\textwidth]{{KG42_1b_vtrap0.005_flux}.eps}
    \caption{低能部分的单粒子态随磁通的周期演化图。图中以42格点的kagom\'{e}格子为例,约束势为$V_{\rm trap}=0.005$。当磁通改变一个周期时,在$l$角动量分区的单粒子态演化到了$l+1 \mod 6$的角动量分区。这里不同颜色的态分别来自不同的初始角动量分区。}
\label{fig:KG42_1b_flux}
\end{figure}

\subsubsection{多体基态的角动量}
依照以上的分析,对于$\frac{1}{2}$分数量子霍尔效应/分数陈绝缘体,体系能量最低的态,也就是基态的构型应该为$|101010 \ldots n_i \ldots \rangle$,其中$\sum_i n_i =N_b $表示系统的粒子数,$n_i = 0, 1$表示占据在第$i$个单粒子态上的玻色子数目。因此,基态的总角动量$L_{\rm GS}^{(N_b)}$可以写成$L_{\rm GS}^{(N_b)} = 5+7+\dots+(2N_b+3) = N_b^2 + 4N_b \mod 6$。计算本文中典型的玻色子占据数情形,可以得到$L_{\rm GS}^{(1)} = 5$, $L_{\rm GS}^{(2)} = 0$, $L_{\rm GS}^{(3)} = 3$, $L_{\rm GS}^{(4)} = 2$, $L_{\rm GS}^{(5)} = 3$, $L_{\rm GS}^{(6)} = 0$。这些从单粒子图像得到的结果与我们的数值结果完全符合(见图~\ref{fig:HC_edge}和图~\ref{fig:KG_edge})。这一结论不依赖与所选的具体晶格结构,也不依赖于系统的具体尺寸。


\subsubsection{多体态的磁通演化周期}
下面具体研究在有限体系中加入中心磁通的情形。上文提到,当中心磁通的强度改变$2\pi$时,每一个单粒子态演化到了相应的下一个角动量区间。对于多体态,其中的每一个玻色子占据在相应的单粒子轨道中。随着磁通周期演化,各个玻色子演化到了下一个角动量区间:对于$N_b$个玻色子的情形,改变$q$个磁通周期时,满足$E_n(L,\Phi+2q\pi)=E_n(L+qN_b,\Phi)$。利用$N_b/6=p/q$可得$E_n(L+qN_b,\Phi)=E_n(L+6p,\Phi)$,其中$p$ 和 $q$为互质整数。考虑到旋转对称性,最终可以得到下列关系式:
\[E_n(L,\Phi+2q\pi)=E_n(L,\Phi)~.\]
这正是之前在中心磁通~\pageref{sec:flux}一节分析数值结果得到的经验公式。


\subsubsection{低能边缘激发的简并序列}
之前的数值结果表明,分数陈绝缘体的边缘激发存在一个准简并的序列谱。前面提到的手征Luttinger液体理论能够很好的解释这一现象,但是这一理论的推导过程并非十分直观。这里,我们试图从另一个角度,即广义Pauli原理,以比较直观的单粒子图像来理解这一简并序列。

讨论这一问题的核心思想是:在遵循广义Pauli原理的前提下,理论上能够列出$N_b$个玻色子在单粒子轨道中所有可能的占据构型,通过统计每个构型的总角动量可以把所有的态进行归类。总角动量区间$L$的简并度应该等于该区间可能存在的占据构型的个数。也就是说,只需要让粒子填充单粒子轨道,并且要求任何连续的两个轨道内粒子数不超过一个,就可以得到相应的简并序列。在实际的严格对角化计算中,所能处理的体系一般较小。在填充数较少的情况下,系统的有限尺寸效应可以忽略,可以期待得到较好的序列谱。

不失一般性,这里以填充$N_b = 3$个玻色子的有限体系为例。这一体系的基态构型应为$|10101000 \ldots \rangle$, 这里 $\ldots$ 表示其余轨道上的占据数,此时均为0。因此,基态的总角动量$L = 5 + 7 + 9 = 21$,或者经过 $L \mod 6$后为3,就如数值计算结果所示。由于基态是满足$L=21$唯一的构型,因此该区间的简并度为$d_{L=21}=1$。在$L=22$的区间,也存在唯一的构型$|10100100 \ldots \rangle$,满足$L=5+7+10 =
22$, 于是$d_{L=22}=1$。然而,在$L=23$的区间,存在两个独立的构型:$|101000100 \ldots \rangle$, 满足 $L=5+7+11 = 23$, 以及 $|10010100 \ldots \rangle$,满足$L=5+8+10 = 23$,因此$d_{L=23}=2$。通过这种方式,原则上可以推断出每一个角动量区间的简并度。在图Fig.~\ref{fig:degeneracycount} 中,我们示意性地列出了$N_b=3$体系的前几个角动量区间的所有独立构型。其中蓝色小短线代表单粒子轨道,红色小球代表填充在轨道上的玻色子。统计构型的总角动量,并将其列在图中左侧。用这种方式,可以很直观地得出简并序列应为$1,1,2,3,4,5,7, ...$,这与数值结果完全符合!

\begin{figure}[!htb]
    \centering
    \includegraphics[width=0.8\textwidth]{degeneracy.eps}
    \caption{体系具有3个硬核玻色子时的单粒子轨道占据构型示意图,其中允许的构型有广义Pauli原理所决定。蓝色短线代表单粒子轨道,红色小球代表占据在该轨道上的玻色子,每一个构型的左侧表明了对应的总角动量量子数。}
\label{fig:degeneracycount}
\end{figure}