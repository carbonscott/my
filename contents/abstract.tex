\begin{abstract}
在凝聚态物理领域,探索与分类各种物质的状态是一项非常引人注目的重要主题。尽管宏观物质都是由原子构成的,在不同的外界条件下,同一种物质会表现出截然不同的结构与性质。朗道的对称破缺理论认为,不同的物质形态对应着不同的原子排列方式,物质形态的改变伴随着体系对称性的破缺。这些不同的物质状态可以用相应的序参量来区分。1980年实验观测到的整数量子霍尔效应,使人们意识到了对称破缺理论并不能完整地描述相变过程。事实上,量子霍尔效应中的不同平台对应着不同的拓扑序,它们之间的相变过程并没有伴随对称性的破缺。之后发现的分数量子霍尔效应更是开拓了探索分数拓扑序的全新领域。到目前为止,所有在实验上观测到的量子霍尔效应都是在极低温,强磁场的朗道能级中观测到的。为了理解量子霍尔效应的内在机理,Haldane在1988年提出了一个具有交错磁通的晶格模型,实现了无朗道能级的整数量子霍尔效应。这一类新颖的晶格体系上的量子霍尔效应,不依赖于外加强磁场,并且具有较大的特征能隙,有可能在常温下存在,现在也称为陈绝缘体。由于Haldane模型的能谱高度色散,在其中分数填充相互作用的粒子时并未观测到分数化现象。

最近的理论研究证实了在无外加磁场的条件下,我们可以在拓扑平坦能带中实现分数量子反常霍尔效应(亦称为分数陈绝缘体)。这一类新近提出的晶格体系模型在探索一大类分数拓扑相中扮演了重要的角色。这里我们开展了系统的数值研究,选取两个具有较大平坦度的拓扑平带模型,观察这类系统(选取有限尺寸的蝶形结构)在填充硬核玻色子情形下的边缘激发行为。边缘态的研究能够为实验中探测拓扑序提供强有力的证据。对于这里研究的六角蜂窝状蝶形结构上的Haldane模型以及kagom\'{e} 蝶形格子模型,我们都观测到了一系列特征边缘激发谱,这与手征Luttinger液体理论对玻色型分数量子霍尔效应的预言符合得相当好。进一步地,我们分别在两种蝶形结构里加入中心磁通,通过调节其强度来检验手征边缘态的载流特性。借助于广义Pauli原理,许多在数值研究过程中获得的无法直观理解的现象,都得到了很好的解释。本文的研究工作直接展示了玻色型分数陈绝缘体的边缘激发特征,证实了具有单位陈数的拓扑平带与朗道能级的拓扑相似性,也为以后进一步的研究高陈数拓扑平带的分数拓扑序铺平了道路。
该硕士论文的部分研究工作已经发表在物理学权威期刊上(Physical Review B: Rapid Communications,第一作者)。


\keywords{量子霍尔效应;拓扑平坦能带;分数陈绝缘体;边缘激发}
%\keywords{\large 量子霍尔效应 \quad Haldane模型 \quad 拓扑平坦能带 \quad 分数陈绝缘体 \quad 边缘激发}
\end{abstract}


\begin{englishabstract}
One of the most essential and fascinating topics in condensed matter physics is to explore and classify the various states of matter, among which the integer quantum Hall effect (IQHE) and the fractional quantum Hall effect (FQHE) have long been the major focus. Despite the transparent fact that all macroscopic systems are made of atoms, even the same matter transforms both its structure and property violently, when such environment, be it temperature or strain, changes. According to Landau's symmetry breaking theory, various states of matter correspond to various ways of atoms' arrangement. The symmetry of arrangement always changes within phase transition, and thus every state can be characterized by some order parameter. When IQHE was first discovered experimentally in 1980, which encounters plateau in Hall conductance whenever filling factor of electrons becomes integer, people started to realize that symmetry breaking is not enough to characterize the course of phase transition. In fact, different plateaus corresponds different kind of topological orders, among which no symmetry is broken. Even more, the discovery of FQHE opens up the gate to investigate fractional topological phases. The experimentally realized QHEs have been so far limited in present with Landau levels, which occurs only under high magnetic field and at extremely low temperature. To understand the deep mechanism in QHE, Haldane proposed a toy model in crystal lattice with staggered flux in 1988, in which he realized IQHE without Landau levels. such newly found IQHEs in lattice system, also dubbed as Chern insulator nowadays, does not impose the existence of uniform magnetic field, and may be observed at normal temperature in the future. When interacting particles are loaded into this energy band, however, no convincing signs of lattice version of FQHE show up.

Recent theoretical works have demonstrated the realization of fractional quantum anomalous Hall states (also called fractional Chern insulators) in topological flat band lattice models without an external magnetic field. Such newly proposed lattice systems play a vital role to obtain a large class of fractional topological phases. Here we report the exact numerical studies of edge excitations for such systems in a disk geometry loaded with hard-core bosons, which will serve as a more viable experimental probe for such topologically ordered states. We find convincing numerical evidence of a series of edge excitations characterized by the chiral Luttinger liquid theory for the bosonic fractional Chern insulators in both the honeycomb disk Haldane model and the kagom\'{e}-lattice disk model. We further verify these current-carrying chiral edge states by inserting a central flux to test their compressibility. With the aid of generalized Pauli principle, we can understand quite well these intriguing phenomena from numerical investigation. This work directly demonstrates edge excitations in fractional Chern insulator, which mimic the edge physics in Laughlin-like states. We hope this work could be a starting point to explore more intriguing phenomena in fractional Chern insulators, including high Chern number band and lattice specific effects. Part of this work has been published on Journal \emph{Physical Review B: Rapid Communications.}

\englishkeywords{quantum Hall effect; topological flat band;
fractional Chern insulator; edge excitations}
\end{englishabstract}
