\chapter{总结与展望}
\label{summary}

本文讨论了两个典型的拓扑平带模型,分别为六角蜂窝格子中的扩展Haldane模型和Kagome格子模型,通过选取适当的参数,可以得到较理想的拓扑平坦能带,所得的平带陈数均为1。在具有六重旋转性的蝶形几何系统中填充硬核玻色子,我们发现了与朗道能级类似的边缘激发谱,可以很好地与描述Laughlin态边缘激发效应的手征Luttinger液体理论相匹配。通过在有限晶格系统中加入中心磁通,并且调节其强度,验证了边缘激发态的可压缩性。边缘激发的研究,有利于将来在光晶格体系中实现分数陈绝缘体/分数量子反常霍尔效应,将成为分析实验现象的强有力的手段。

本文讨论了两个典型的拓扑平带模型,通过系统研究其有限尺寸的蝶形几何系统,来探索分数陈绝缘体中的边缘激发效应。在外加一个对称的中心谐振势的条件下,我们观察到了明显的低能边缘激发谱,得到的数值结果与手征Luttinger液体理论符合得非常好。通过在有限晶格系统中加入中心磁通,并且调节其强度,我们检验了边缘激发态的可压缩性。我们期待边缘激发谱的研究成为揭示体拓扑序提供一种有力的手段,并且成为未来在实验中观测分数陈绝缘体/分数量子反常霍尔效应的强有力的证据。
除此之外,我们发现广义Pauli原理能够很好的理解得到的数值结果,这对我们将来分析更复杂情形的分数陈绝缘体提供了很好的帮助。比如高陈数的分数陈绝缘体、非阿贝尔型分数陈绝缘体。
