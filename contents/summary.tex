\chapter{总结与展望}
\label{summary}

%本文讨论了两个典型的拓扑平带模型,分别为六角蜂窝格子中的扩展Haldane模型和Kagome格子模型,通过选取适当的参数,可以得到较理想的拓扑平坦能带,所得的平带陈数均为1。在具有六重旋转性的碟形几何系统中填充硬核玻色子,我们发现了与朗道能级类似的边缘激发谱,可以很好地与描述Laughlin态边缘激发效应的手征Luttinger液体理论相匹配。通过在有限晶格系统中加入中心磁通,并且调节其强度,验证了边缘激发态的可压缩性。边缘激发的研究,有利于将来在光晶格体系中实现分数陈绝缘体/分数量子反常霍尔效应,将成为分析实验现象的强有力的手段。

本文讨论了两个典型的拓扑平带模型,分别为六角蜂窝格子中的扩展Haldane模型和Kagome格子模型,通过系统研究其有限尺寸的碟形几何系统,来探索分数陈绝缘体中的边缘激发效应。在外加一个对称的中心谐振势的条件下,我们观察到了明显的低能边缘激发谱,得到的数值结果与手征Luttinger液体理论符合得非常好。除此之外,我们发现广义Pauli原理能够很好的理解得到的数值结果,这对我们将来分析更复杂情形的分数陈绝缘体提供了很好的帮助。比如高陈数的分数陈绝缘体、非阿贝尔型分数陈绝缘体。通过在有限晶格系统中加入中心磁通,并且调节其强度,我们检验了边缘激发态的载流特性。我们期待边缘激发谱的研究成为揭示体拓扑序提供一种有力的手段,并且成为未来在实验中观测分数陈绝缘体/分数量子反常霍尔效应的强有力的证据。

基于本文的研究工作,将来可以在以下几个方面做进一步的研究。对于朗道能级中的分数量子霍尔态,我们有解析的模型波函数:Laughlin态,通过解析波函数可以更好地解释霍尔态的本质特征。同样地,对于连续模型在晶格体系的推广,我们是否能够写出碟形几何结构中分数陈绝缘体的解析波函数,借此对这一分数拓扑序有更好地理解。另一方面,这里研究的单位陈数的量子反常霍尔效应,与连续模型中的朗道能级可以相互对于,它们都具有单位陈数;而在陈绝缘体中理论上可以得到任意值的陈数,对于无连续对应的高陈数体系,又会出现何种方程的分数拓扑序。除此之外,非阿贝尔型量子反常霍尔效应的边缘激发也非常值得研究,这有可能在将来的拓扑量子计算中得到应用。

